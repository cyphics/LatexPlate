\usepackage{microtype}



% =================================
% ENCODING
% =================================

\usepackage[francais]{babel}
%\usepackage{fontspec}

\usepackage[T1]{fontenc}
\usepackage[utf8]{inputenc}
\usepackage[hyphens]{url}
% For japanese font
\usepackage{xeCJK}
\setCJKmainfont{IPAMincho} % for \rmfamily
%\DeclareUnicodeCharacter{00A0}{ } % Semble servir à identifier
% l'erreur lorsqu'un caractère ne respecte pas UTF-8

% =================================
% LAYOUT GENERAL
% =================================

\usepackage{parskip}
\setlength{\parskip}{.5em}

% =================================
% TABLE DES MATIERES
% =================================

% redefine space after part number in toc
\renewcommand\partnumberlinebox[2]{#2\hspace{1em}}
\setcounter{tocdepth}{2} % set Table of Contents depth
% Reset chapter number for every part
\makeatletter
\@addtoreset{chapter}{part}
\makeatother

% =================================
% BIBLIO
% =================================
\usepackage[backend=biber,style=authoryear,doi=false,isbn=false,url=false,firstinits=true]{biblatex}
\addbibresource{These.bib}
%\nocite{*}
\AtEveryBibitem{%
  \clearfield{note}%
  }


% =================================
% LINGUISTIQUE
% =================================

\usepackage{linguex}

\newcommand{\guill}[1]{\guillemotleft#1\guillemotright}
\newcommand\cad[1]{\textsc{#1}*}
\newcommand\cah{concepts \emph{ad hoc}}
\newcommand\tcah{théorie des \cah}
\newcommand\mention[1]{`#1'}
\newcommand\mot[1]{\emph{#1}}
\newcommand{\concept}[1]{\emph{#1}}
\newcommand{\boiteAExemples}[2]{\textbf{#1 :}\\\textit{#2}}
\newcommand\explicitation[1]{``#1''}


% =================================
% CITATIONS
% =================================

%\usepackage[svgnames]{xcolor}
\usepackage[table,svgnames]{xcolor}
\usepackage{csquotes}
\usepackage{tikz}
\usepackage{framed}
\usepackage{mdframed}
% \usepackage{etoolbox}
\newcommand\reprise[1]{``#1''}
\newcommand\auteur[1]{\textsc{#1}}
\newcommand\enonce[1]{``#1''}
\newcommand*\quotefont{\fontfamily{LinuxLibertineT-LF}}
% relative

%\input{macro_shadequote}

\usepackage{hyperref}
\usepackage{xcolor}
\hypersetup{
    colorlinks,
    linkcolor={blue!40!black},
    citecolor={blue!50!black},
    urlcolor={blue!80!black}
}
%%% Local Variables:
%%% mode: latex
%%% TeX-master:"these"
%%% End:
